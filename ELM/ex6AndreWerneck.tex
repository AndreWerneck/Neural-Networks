\documentclass{article}

\usepackage{Sweave}
\begin{document}
\input{ex6AndreWerneck-concordance}

\title{Exercicio 6 - Redes Neurais Artificiais}
\author{Andre Costa Werneck}
\date{16/05/2022}
\maketitle
\newpage






\section{ELM}
\subsection{BASE BREAST CANCER}

Para a base do Breast Cancer, seguem as acuracias com:
\\p = 5 

\begin{Schunk}
\begin{Soutput}
acuracia treino = 0.829447852760736 +- 0.0481305989626327
\end{Soutput}
\begin{Soutput}
acuracia teste = 0.913875598086124
\end{Soutput}
\end{Schunk}

p = 10 


\begin{Schunk}
\begin{Soutput}
acuracia treino = 0.891104294478528 +- 0.0430797526495011
\end{Soutput}
\begin{Soutput}
acuracia teste = 0.933014354066986
\end{Soutput}
\end{Schunk}

p = 30 


\begin{Schunk}
\begin{Soutput}
acuracia treino = 0.945705521472393 +- 0.00819273031656856
\end{Soutput}
\begin{Soutput}
acuracia teste = 0.923444976076555
\end{Soutput}
\end{Schunk}

p = 50 


\begin{Schunk}
\begin{Soutput}
acuracia treino = 0.96359918200409 +- 0.00595657835650396
\end{Soutput}
\begin{Soutput}
acuracia teste = 0.956937799043062
\end{Soutput}
\end{Schunk}

p = 100 


\begin{Schunk}
\begin{Soutput}
acuracia treino = 0.980470347648262 +- 0.00457153264031039
\end{Soutput}
\begin{Soutput}
acuracia teste = 0.92822966507177
\end{Soutput}
\end{Schunk}

p = 300 


\begin{Schunk}
\begin{Soutput}
acuracia treino = 1 +- 0
\end{Soutput}
\begin{Soutput}
acuracia teste = 0.84688995215311
\end{Soutput}
\end{Schunk}

Para comecar a analise, vale ressaltar que para cada treinamento da rede, com cada numero diferente de neuronios, foi feito um loop for que repetiu o treinamento e pegou os valores medios da acuracia, dos pesos e de Z, como pedido. 

Dessa forma, observou-se, claramente, que para o conjunto de treinamento, a convergencia do modelo aumenta a medida que o numero de neuronios tambem cresce. Entretanto, a acuracia de teste parece ser maxima com o valor de p=30 ou p=50, nesse caso. Foi muito interessante notar, ademais, que se aumentarmos muito o numero de neuronios, a acuracia do conjunto de teste cai consideravelmente, mesmo que a acuracia do conjunto de treino permaneca maxima (em 100\%). Isso e um sinal claro de perda de capacidade de generalizacao do modelo, ou seja, e um forte indicio de overfitting. Dessa forma, foi valido observar que nem sempre a maxima acuracia de treino representa a melhor solucao para o problema, ilustrando ponto que ja haviamos aprendido nas aulas da disciplina. 

Vale ressaltar que um treinamento com a mesma base de dados foi realizado com um Perceptron simples no exercicio anterior. No caso do perceptron, houve boa convergencia com um numero de epocas a partir de 100, com acuracia chegando a 97,6\%. Comparando os dois modelos vale dizer que foi observada uma maior rapidez na ELM para uma acuracia muito semelhante (com p= 30 e 50). Dessa forma, creio que fica evidente a melhora gracas a linearizacao do modelo advinda da camada escondida da ELM, modelo que pareceu apresentar, para esse, problema, menor custo computacional.   

\subsection{Statlog (Heart)}




Para a base da Statlog, seguem as acuracias com:
\\p = 5 

\begin{Schunk}
\begin{Soutput}
acuracia treino = 0.726719576719577 +- 0.0245948638958122
\end{Soutput}
\begin{Soutput}
acuracia teste = 0.753086419753086
\end{Soutput}
\end{Schunk}

p = 10 


\begin{Schunk}
\begin{Soutput}
acuracia treino = 0.828306878306878 +- 0.0320451424554358
\end{Soutput}
\begin{Soutput}
acuracia teste = 0.839506172839506
\end{Soutput}
\end{Schunk}

p = 30 


\begin{Schunk}
\begin{Soutput}
acuracia treino = 0.864814814814815 +- 0.00795505120462764
\end{Soutput}
\begin{Soutput}
acuracia teste = 0.827160493827161
\end{Soutput}
\end{Schunk}

p = 50 


\begin{Schunk}
\begin{Soutput}
acuracia treino = 0.883068783068783 +- 0.00839219069353207
\end{Soutput}
\begin{Soutput}
acuracia teste = 0.839506172839506
\end{Soutput}
\end{Schunk}

p = 100 


\begin{Schunk}
\begin{Soutput}
acuracia treino = 0.946031746031746 +- 0.0135819947978325
\end{Soutput}
\begin{Soutput}
acuracia teste = 0.703703703703704
\end{Soutput}
\end{Schunk}

p = 300 


\begin{Schunk}
\begin{Soutput}
acuracia treino = 1 +- 0
\end{Soutput}
\begin{Soutput}
acuracia teste = 0.555555555555556
\end{Soutput}
\end{Schunk}

Oservou-se que a mesma analise da base do breast cancer pode, tambem, ser feita para a base da Stalog. Vale apenas ressaltar que o modelo teve muito mais dificuldade de aprendizado para a presente base. Foi necessaria uma normalizacao dos dados para ajudar na obtencao de uma solucao melhor e, mesmo convergindo ate 100\% no treinamento, para os conjuntos de teste, a acuracia maxima nao passou muito dos 85\%. No que concerne overfitting e acuracias maximas, a analise e identica a da base do breast cancer.

\section{Perceptron - Statlog(heart)}



